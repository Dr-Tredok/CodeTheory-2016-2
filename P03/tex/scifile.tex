% Use only LaTeX2e, calling the article.cls class and 12-point type.

\documentclass[12pt]{article}

% Users of the {thebibliography} environment or BibTeX should use the
% scicite.sty package, downloadable from *Science* at
% www.sciencemag.org/about/authors/prep/TeX_help/ .
% This package should properly format in-text
% reference calls and reference-list numbers.

\usepackage{scicite}

% Use times if you have the font installed; otherwise, comment out the
% following line.

\usepackage{times}

% The preamble here sets up a lot of new/revised commands and
% environments.  It's annoying, but please do *not* try to strip these
% out into a separate .sty file (which could lead to the loss of some
% information when we convert the file to other formats).  Instead, keep
% them in the preamble of your main LaTeX source file.


% The following parameters seem to provide a reasonable page setup.

\topmargin 0.0cm
\oddsidemargin 0.2cm
\textwidth 16cm 
\textheight 21cm
\footskip 1.0cm


%The next command sets up an environment for the abstract to your paper.

\newenvironment{sciabstract}{%
\begin{quote} }
{\end{quote}}


% If your reference list includes text notes as well as references,
% include the following line; otherwise, comment it out.

\renewcommand\refname{Bibliograf\'ia}

% The following lines set up an environment for the last note in the
% reference list, which commonly includes acknowledgments of funding,
% help, etc.  It's intended for users of BibTeX or the {thebibliography}
% environment.  Users who are hand-coding their references at the end
% using a list environment such as {enumerate} can simply add another
% item at the end, and it will be numbered automatically.

\newcounter{lastnote}
\newenvironment{scilastnote}{%
\setcounter{lastnote}{\value{enumiv}}%
\addtocounter{lastnote}{+1}%
\begin{list}%
{\arabic{lastnote}.}
{\setlength{\leftmargin}{.22in}}
{\setlength{\labelsep}{.5em}}}
{\end{list}}

\usepackage{amsmath}
\usepackage[utf8]{inputenc}
\usepackage{hyperref}

% Include your paper's title here

\title{Reed-Solomon\\\small{Práctica 3}} 


% Place the author information here.  Please hand-code the contact
% information and notecalls; do *not* use \footnote commands.  Let the
% author contact information appear immediately below the author names
% as shown.  We would also prefer that you don't change the type-size
% settings shown here.

\author
{\small{\textsc{Código}} -at- \href{https://github.com/Dr-Tredok/CodeTheory-2016-2/tree/master/P03}{GitHub}}

% Include the date command, but leave its argument blank.

\date{}



%%%%%%%%%%%%%%%%% END OF PREAMBLE %%%%%%%%%%%%%%%%



\begin{document} 

% Double-space the manuscript.

\baselineskip24pt

% Make the title.

\maketitle 



% Place your abstract within the special {sciabstract} environment.

\begin{sciabstract}
  Codificación de $RS(255, 223)$ usando el polinomio:
  \begin{align*}
  g(x) = (x - \alpha )(x -\alpha^2)\ldots (x - \alpha^{32}) \in GF(256)[x]
  \end{align*}
  $RS(255, 223)$ es un $[n = 255, k = 223, d = 33]_{256} -$ código sobre $GF(256)$ con $\alpha$ un elemento primitivo.
  \\\textit{Obs.} $deg(g) = 32 = n - k = d - 1 = 2 t = 2 \lfloor \frac{d - 1}{2} \rfloor$, i.e. corrige hasta $ t = 16$ errores. Además $g(x)$ es un factor de $x^n - 1$ por lo que se trata de un código cíclico. 
\end{sciabstract}



% In setting up this template for *Science* papers, we've used both
% the \section* command and the \paragraph* command for topical
% divisions.  Which you use will of course depend on the type of paper
% you're writing.  Review Articles tend to have displayed headings, for
% which \section* is more appropriate; Research Articles, when they have
% formal topical divisions at all, tend to signal them with bold text
% that runs into the paragraph, for which \paragraph* is the right
% choice.  Either way, use the asterisk (*) modifier, as shown, to
% suppress numbering.

\section*{GF(256)}

Formalmente,
\begin{align*}
GF(256) = \mathrm{F}_{2^8} = \mathrm{Z}_2[x] / <f(x)> \text{ donde } deg(f) = 8
\end{align*}
El campo puede ser generado con los siguientes polinomios (entre otros):
\begin{itemize}
\item \textbf{AES} $x^8 + x^4 + x^3 + x + 1$ con $\alpha = x + 1$ como raíz primitiva
\item \textbf{Primitivo} $x^8 + x^4 + x^3 + x^2 + 1$ con $\alpha = x$ como raíz primitiva
\end{itemize}
Se representa un elemento en $GF(256)$ con un byte:
\begin{align*}
b_7b_6\ldots b_0 \mapsto b(x) = b_7 x^7 + b_6 x^6 + \ldots + b_1 x + b_0
\end{align*}

\section{Codificación}
El código $RS(255, 223)$ toma bloques de 223 bytes y los codifica a un bloque de 255, 
sea $m = (m_{k-1}, \ldots , m_1, m_0)\in F_{256}^k$ el mensaje a codificar y $m(x)$ el polinomio asociado:
\begin{align*}
m(x) = m_{k-1} x^{k-1} + \ldots + m_1 x + m_0
\end{align*}
Para codificar usando el polinomio $g(x)$
\begin{enumerate}
\item $s(x) = m(x) x^{32} = q(x) g(x) + p(x)$ donde $deg(s) < k - 1 + n - k = 254$ y $deg(p) < 32$
\begin{align*}
s(x) &= m_{222} x^{254} + \ldots + m_1 x^{33} + m_0 x^{32}
\\p(x) &= p_{31} x^{31} + \ldots + p_1 x + p_0
\end{align*}
\item La palabra codificada $c(x) = s(x) - p(x) = q(x)g(x)$:
\begin{align*}
c(x) &= m_{222} x^{254} + \ldots + m_1 x^{33} + m_0 x^{32} + p_{31} x^{31} + \ldots + p_1 x + p_0
\\c &= (m_{222}, \ldots , m_1, m_0, p_{31}, \ldots , p_1, p_0) \in F_{256}^n
\end{align*}
El polinomio codificado $c(x)$ contiene al mensaje $m(x)$ (Codificación Sistemática) y $2t$ entradas de verificación.
\end{enumerate}
Ya que $c(x) | g(x)$ ent $g(\alpha^i) = 0$ syss $c(\alpha^i) = 0$, donde $i = 1..2t$. Se tiene entonces:
\begin{align*}
c(x) \in RS(255, 223) \Leftrightarrow c(\alpha^i) = 0, \forall i = 1..2t
\end{align*}


\section{Decodificación}
Sea $r(x) = c(x) + e(x)$ el mensaje recibido, donde $e(x) = e_{n-1} x^{n-1} + \ldots + e_1 x + e_0$ es el polinomio de errores en $r(x)$, i.e. $e_i$ es el error en la posición $i$.

\textit{Obs.} Si $\omega (e) > t$ entonces la capacidad de corrección es excedida.

\subsection{Síndromes}
Definimos $s_i$ para $i = 1...2t$ como el síndrome para $\alpha_i$:
\begin{align*}
r(x) = q_i(x)(x - \alpha^i) + s_i(x)
&\Leftrightarrow
s_i(x) = q_i(x)(x - \alpha^i) + r(x)
\\\text{evaluando en } & \alpha^i:
\\s_i(\alpha^i) &= q_i(x)(\alpha^i - \alpha^i) + r(\alpha^i)
\\&= r_{n-1}(\alpha^i)^{n-1} + \ldots + r_1 \alpha^i + r_0
\end{align*}
i.e. basta evaluar las potencias de $\alpha$ en $r(x)$ para encontrar los síndromes.

Por otro lado,
\begin{align*}
r(\alpha^i) &= c(\alpha^i) + e(\alpha^i)
\\ &= e(\alpha^i) = s_i
\end{align*}
por lo que si todos los síndromes son cero entonces no hay errores y $r(x) = c(x)$.

\subsection{Localización de errores}
Asumiendo que $v \leq t$ errores ocurren, podemos escribir:
\begin{align*}
e(x) = Y_1 x^{e_1} + \ldots + Y_v x^{e_v}
\end{align*}
donde $Y_1, \ldots , Y_v$ son los valores de los errores y $e_1, \ldots , e_v$ son las posiciones.

Sabemos que $s_i = e(\alpha_i)$, por lo que se tiene:
\begin{align*}
s_i &= Y_1 (\alpha^i)^{e_1} + \ldots + Y_v (\alpha^i)^{e_v}
\\ &= Y_1 X_1^i + \ldots + Y_v X_v^i
\end{align*}
A los términos $X_j = \alpha^{e_j}$ se les denomina localizadores de errores.

Ya que las potencias de $\alpha$ son distintas, si se conocen los localizadores $X_j$ se pueden determinar las posiciones de los errores. Para ello se define $S(x) = s_1 + s_2 x + \ldots + s_{32} x^{31}$ y $\sigma (x)$ (localizador de errores):
\begin{align*} 
\sigma (x) &= \prod_{i=1}^v (1 - X_ix) 
\\ &=\sum_{i = 0}^v \sigma_i x^i = 1 + \Lambda_1 x + \ldots + \Lambda_v x^v
\end{align*}
\textit{Obs. } Las raíces de $\sigma (x)$ son los inversos de los localizadores $X_j$. 

Sea $\omega(x)$\footnote{Polinomio evaluador de errores} tal que $deg(\omega ) < t$ se obtiene la ecuación:
\begin{align*}
\omega (x) &\equiv \sigma (x) S(x)\mod x^{32}
\\\Leftrightarrow \omega (x) &= \theta (x) x^{32} + \sigma (x) S(x)
\end{align*}
usando el algoritmo de Euclides para $x^{32}$ y $S(x)$ se obtienen residuos de la forma:
\begin{align*}
r_k(x) = a_k(x) x^{32} + b_k(x) S(x)
\end{align*}
cuando $deg(r_k) < 16$ pero $deg(r_{k-1}) \geq 16$ y $deg(b_k) \leq 16$ se cumple:
\begin{align*}
\sigma (x) &= b_k(0)^{-1}b_k(x) & \omega (x) &= b_k(0)^{-1}r_k(x)
\end{align*}
de donde se obtienen los localizadores de errores $X_1 = \alpha^{e_1}, \ldots , X_v  = \alpha^{e_v}$.
\section{Corrección de errores}
Siguiendo el algoritmo de Forney, dado $\sigma (x)$ el polinomio localizador de errores, se define:
\begin{align*}
\sigma '(x) &= \sum_{i=1}^v i \cdot \Lambda_i x^{i-1}
\end{align*}
Sean $e_1, \ldots , e_v$ las posiciones de los errores, para $i=1...v$ la magnitud del error $e_i$:
\begin{align*}
Y_i &= -\frac{\omega (X_i^{-1})}{\sigma '(X_i^{-1})}
\end{align*}

Entonces, la decodificación resulta:
\begin{align*}
r(x) &= c(x) + e(x)
\\c(x) &= r(x) + e(x)
\\ &= r(x) + Y_1x^{e_1} + \ldots + Y_vx^{e_v}
\end{align*}


\section{Ejemplo}\footnote{Código para este ejemplo en \texttt{test.py}}
En $GF(16)$ se define $RS(15, 11)$ con capacidad de correción $2$.
Sea $c(x) = x^8 + x^7 + x^6 + x^4 + 1$ el mensaje enviado y $r(x) = x^{12} + x^8 + x^7  + x^6 + 1$ el mensaje recibido.
Los síndromes:
\begin{align*}
s_1 &= r(\alpha) = \alpha^6
\\s_2 &= r(\alpha^2) = \alpha^{12}
\\s_3 &= r(\alpha^3) = \alpha^4
\\s_4 &= r(\alpha^4) = \alpha^9
\end{align*}
Se aplica el algoritmo de euclides para $S(x) = \alpha^9 x^3 + \alpha^4 x^2 + \alpha^{12}x + \alpha^6$ y $x^4$:
\begin{align*}
r_k(x) &= a_k(x) x^4 + b_k(x) S(x)
\\\alpha^9 &= (\alpha^{13} x + \alpha^{13})x^4 + (\alpha^4 x^2+ \alpha^9 x+ \alpha^3 ) S(x)
\end{align*}
por lo que se tiene:
\begin{align*}
\sigma (x) &= \alpha^{11} b_k(x) =\alpha x^2 + \alpha^6 x + 1
\\\omega (x) &= \alpha^{11} r_k(x) = \alpha^6
\end{align*}

Se encuentran las raíces de $\sigma (x)$:
\begin{align*}
\sigma (\alpha^3) &= \alpha\alpha^6 + \alpha^6\alpha^3 + 1 = \alpha^7 + \alpha^9 + 1  
\\ &= (x^3 + x + 1) + (x^3 + x) + 1 = 0
\\\sigma(\alpha^{11}) &= \alpha\alpha^{22 \mod 15} + \alpha^6\alpha^{11} + 1
\\ &= (x^2 + 1) + x^2 + 1 = 0
\end{align*}
por lo que los inversos, son:
\begin{align*}
X_1 &= \alpha^{12} = (\alpha^3)^{-1}
\\ X_2 &= \alpha^4 = (\alpha^{11})^{-1}
\end{align*}
es decir, hay errores en la posición $e_1 = 12$ y $e_2 = 4$. Para la correción se obtiene:
\begin{align*}
\sigma '(x) &= \alpha^6
\\Y_i &= - \frac{\omega(X_i^{-1})}{\sigma '(X_i^{-1})}
\\ &= - \frac{\alpha^6}{\alpha^6} = 1
\end{align*}
Se obtiene:
\begin{align*}
e(x) &= x^{e_1} + x^{e_2} = x^{12} + x^4
\\ c(x) &=  x^{12} + x^8 + x^7  + x^6 + 1 + x^{12} + x^4
\\ &= x^8 + x^7 + x^6 + x^4 + 1
\end{align*}
% Your references go at the end of the main text, and before the
% figures.  For this document we've used BibTeX, the .bib file
% scibib.bib, and the .bst file Science.bst.  The package scicite.sty
% was included to format the reference numbers according to *Science*
% style.


\begin{thebibliography}{1}
\bibitem{roman}
  Roman, Steven.
  \emph{Coding and Information Theory}.
  Springer-Verlag,
  1992
\end{thebibliography}


\end{document}




















